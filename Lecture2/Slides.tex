\documentclass[10pt]{beamer}

\input ../pack.tex
\input ../defs.tex
\input ../form.tex


\title{\large \bfseries Stats 205: \\ Introduction to Nonparametric Statistics \linebreak \linebreak \linebreak
Lecture 2: \\ Enumerations}

\author{Instructor: Christof Seiler}

\date{Spring 2016}

\begin{document}

\frame{
\thispagestyle{empty}
\titlepage
}

\begin{frame}
\frametitle{Last Lecture}

In our last lecture, we saw two examples of nonparametric statistics in action. \newline

With a ranked-based method, we tested whether social awareness is improved when sending kids to school versus home schooling.
We assumed that the error was independent and identically distributed (iid) and symmetric around $0$ but made no futher shape assumptions. \newline

In a second example, we saw how to calculate a confidence interval around an estimate using the bootstrap procedure. We used this to evaluate whether aspirin helps reduce the risk of heart attacks in middle-aged men. 
Again, we did not make any assumption about the distribution the noise that corrupted the underlying parameter. 

\end{frame}

\begin{frame}
\frametitle{Today}

We will take a closer look at the computational tools that are needed to for both examples. In both example, we need to somehow describe all possible options that a parameter can take and then compare this distribution of outcomes to the observed outcome to judge how probable it is to observe it. \newline

Most of the time it is not possible to enumerate all possible options, so we need ways to approximate it. Some ways are:
\begin{itemize}
\item Monte Carlo
\item Markov chain Monte Carlo
\end{itemize}

\vspace{0.3cm}
Today, to get a feeling for this, we focus on a special case where we can actually enumerate all options using Gray Codes.

\end{frame}

\begin{frame}
\frametitle{Gray Codes}

Gray codes are ordered lists of binary $n$-tuples. \newline

They are ordered so that success values only differ in a single space.

For instance, for $n = 3$, the list of is:
\[ 000, 001, 011, 010, 110, 111, 101, 100 \]

Notice, a computer scientist might intuitively want to write this:
\[ 000, 001, \cancel{010}, 011, \cancel{100}, 101, \cancel{110}, 111 \]
This is wrong. \newline

A better than trying to reorder the wrong elements, we can define recursive algorithm to generate a valid list.

\end{frame}

\begin{frame}
\frametitle{Gray Codes}

\begin{itemize}
\item Start with list for $n = 1$, which is just $0$ and $1$ 
\item Get two list by putting a zero before each entry and a one before each entry in $L_n$ 
\item To get $L_{n+1}$ concatenate the two list by first followed by second in reversed order 
\end{itemize}

\vspace{0.3cm}
So from $0,1$, we get two lists
\[ 00, 01 \]
\[ 10, 11 \]
and concatenate
\[  00, 01, 11, 10 \]

\end{frame}

\begin{frame}
\frametitle{Next Lecture}

In the next lecture, we will focus on
\begin{itemize}
\item rank-based methods
\end{itemize}

\end{frame}

\end{document}
