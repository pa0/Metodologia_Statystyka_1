\documentclass[10pt]{beamer}

\usepackage{graphicx,amsmath,amssymb,tikz,psfrag}
\usepackage{hyperref}

\input ../defs.tex

%% formatting

\mode<presentation>
{
\usetheme{default}
}
\setbeamertemplate{navigation symbols}{}
\usecolortheme[rgb={0.13,0.28,0.59}]{structure}
\setbeamertemplate{itemize subitem}{--}
\setbeamertemplate{frametitle} {
	\begin{center}
	  {\large\bf \insertframetitle}
	\end{center}
}

\newcommand\footlineon{
  \setbeamertemplate{footline} {
    \begin{beamercolorbox}[ht=2.5ex,dp=1.125ex,leftskip=.8cm,rightskip=.6cm]{structure}
      \footnotesize %\insertsection
      \hfill
      {\insertframenumber}
    \end{beamercolorbox}
    \vskip 0.45cm
  }
}
\footlineon

\AtBeginSection[] 
{ 
	\begin{frame}<beamer> 
		\frametitle{Outline} 
		\tableofcontents[currentsection,currentsubsection] 
	\end{frame} 
} 

\setbeamercolor{alerted text}{fg=blue} 

%% begin presentation

\title{\large \bfseries Stats 205: \\ Introduction to Nonparametric Statistics \linebreak \linebreak \linebreak
Day 1: \\ Logistics and Introduction}

\author{Instructor: Christof Seiler}

\date{Spring 2016}

\begin{document}

\frame{
\thispagestyle{empty}
\titlepage
}

\begin{frame}
\frametitle{Course Website}

You can find everything on our course website: \linebreak

\url{http://christofseiler.github.io/stats205/}

\end{frame}

\begin{frame}
\frametitle{Today}

\begin{itemize}
\item Why?
\item Goals
\item Textbooks
\item Grading
\item History
\item Neuroimaging example
\end{itemize}

\end{frame}

\begin{frame}
\frametitle{Why?}

\begin{enumerate}
\item Few assumptions about underlying populations from which data is obtained, e.g. populations don't need to follow a normal distribution
\item Often easier to understand and apply than parametric tests
\item Slightly less efficient than parametric test when parametric assumptions hold, but if assumptions don't hold then wildly more efficient
\item Jackknife and bootstrap can be used in many practical situations where theory is intractable
\item Bayesian methods are available so prior information can be incorporated
\end{enumerate}

\end{frame}

\begin{frame}
\frametitle{Goals}

\begin{itemize}
\item Get an overview of classical and modern methods
\item Learn how to implement methods yourself and use existing R packages
\item Be aware and understand underlying assumptions
\item Apply to modern data analysis problems that you care about
\end{itemize}

\end{frame}

\begin{frame}
\frametitle{Textbooks}

Our main textbook with lots of practical computations in R (free online): \newline
Kloke and McKean (2015). Nonparametric Statistical Methods Using R \newline

Bayesian view (free online): \newline
M\"uller, Quintana, Jara, and Hanson (2015). Bayesian Nonparametric Data Analysis \newline

In-depth coverage of the bootstrap: \newline
Efron and Tibshirani (1994). An Introduction to the Bootstrap  \newline

Very comprehensive (free online): \newline
Hollander and Wolfe, and Chicken (2013). Nonparametric Statistical Methods (3rd Edition). \newline

In-depth mathematical account on rank-based methods: \newline
Lehmann (2006). Nonparametrics Statistical Methods Based on Ranks \newline

\end{frame}

\begin{frame}
\frametitle{Grading: Homework and Projects}

\begin{itemize}
\item Bi-weekly homework assigments (40\%), mostly R exercises
\item Project (60\%)
\end{itemize}

The overall goal of the project is to write a paper on applying nonparametric statistics to your field of interest or studying some theoretical aspects that your care about.

\end{frame}

\begin{frame}
\frametitle{Grading: Details on Projects}

The project will be split in two parts:
\begin{enumerate}
\item Midterm project (30\%): Project proposal and outline of planned tasks (5 pages)
\item Final project (30\%): 
\begin{itemize}
\item A theoretical part: Explanation of the method studied and its properties
\item A computational part: Preferably in R
\item A data-analysis part: Plots, $p$-values, and interpretations
\end{itemize}
\end{enumerate}

\end{frame}

\begin{frame}
\frametitle{History}

\end{frame}

\begin{frame}
\frametitle{Neuroimaging Example}

\end{frame}

\end{document}
